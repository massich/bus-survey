%% This is file `elsarticle-template-2-harv.tex',
%%
%% Copyright 2009 Elsevier Ltd
%%
%% This file is part of the 'Elsarticle Bundle'.
%% ---------------------------------------------
%%
%% It may be distributed under the conditions of the LaTeX Project Public
%% License, either version 1.2 of this license or (at your option) any
%% later version.  The latest version of this license is in
%%    http://www.latex-project.org/lppl.txt
%% and version 1.2 or later is part of all distributions of LaTeX
%% version 1999/12/01 or later.
%%
%% The list of all files belonging to the 'Elsarticle Bundle' is
%% given in the file `manifest.txt'.
%%
%% Template article for Elsevier's document class `elsarticle'
%% with harvard style bibliographic references
%%
%% $Id: elsarticle-template-2-harv.tex 155 2009-10-08 05:35:05Z rishi $
%% $URL: http://lenova.river-valley.com/svn/elsbst/trunk/elsarticle-template-2-harv.tex $
%%
\documentclass[authoryear,preprint,review,12pt]{elsarticle}

%% Use the option review to obtain double line spacing
%% \documentclass[authoryear,preprint,review,12pt]{elsarticle}

%% Use the options 1p,twocolumn; 3p; 3p,twocolumn; 5p; or 5p,twocolumn
%% for a journal layout:
%% \documentclass[final,authoryear,1p,times]{elsarticle}
%% \documentclass[final,authoryear,1p,times,twocolumn]{elsarticle}
%% \documentclass[final,authoryear,3p,times]{elsarticle}
%% \documentclass[final,authoryear,3p,times,twocolumn]{elsarticle}
%% \documentclass[final,authoryear,5p,times]{elsarticle}
%% \documentclass[final,authoryear,5p,times,twocolumn]{elsarticle}

%% if you use PostScript figures in your article
%% use the graphics package for simple commands
%% \usepackage{graphics}
%% or use the graphicx package for more complicated commands
%% \usepackage{graphicx}
%% or use the epsfig package if you prefer to use the old commands
%% \usepackage{epsfig}

%% The amssymb package provides various useful mathematical symbols
\usepackage{amssymb,amstext}
%% The amsthm package provides extended theorem environments
%% \usepackage{amsthm}

%% The lineno packages adds line numbers. Start line numbering with
%% \begin{linenumbers}, end it with \end{linenumbers}. Or switch it on
%% for the whole article with \linenumbers after \end{frontmatter}.
\usepackage{lineno}

%% natbib.sty is loaded by default. However, natbib options can be
%% provided with \biboptions{...} command. Following options are
%% valid:

%%   round  -  round parentheses are used (default)
%%   square -  square brackets are used   [option]
%%   curly  -  curly braces are used      {option}
%%   angle  -  angle brackets are used    <option>
%%   semicolon  -  multiple citations separated by semi-colon (default)
%%   colon  - same as semicolon, an earlier confusion
%%   comma  -  separated by comma
%%   authoryear - selects author-year citations (default)
%%   numbers-  selects numerical citations
%%   super  -  numerical citations as superscripts
%%   sort   -  sorts multiple citations according to order in ref. list
%%   sort&compress   -  like sort, but also compresses numerical citations
%%   compress - compresses without sorting
%%   longnamesfirst  -  makes first citation full author list
%%
%% \biboptions{longnamesfirst,comma}

\biboptions{numbers,square}

%Sik´s packages load.
\usepackage[nolist]{acronym}
\usepackage{tikz,xifthen}
\usepackage{epsf,graphicx,subfig}
\usepackage{colortbl}
\usepackage{xcolor}
\usepackage{scalefnt,lmodern}
\usepackage{booktabs, multicol, multirow}
\usetikzlibrary{arrows,calc,positioning,backgrounds,snakes,decorations.text,decorations.markings,shapes,patterns}

%\usepackage[backend=natbib]{biblatex}
%\addbibresource{bibliografiaNew.bib}

\journal{Medical Image Analysis}
\newcommand*{\ch}{\checkmark}

\begin{document}

\definecolor{autoGuided}{rgb}{ 0.3765    0.7294    0.9412}
\newcommand{\autoGuidedColor}{(light-Blue)}
\definecolor{fullyAuto}{rgb}{ 0.0941    0.3843    0.6627}
\newcommand{\fullyAutoColor}{(dark-blue)}
\definecolor{semiAuto}{rgb}{ 0.0784    0.5059    0.1686}
\newcommand{\semiAutoColor}{(light-green)}
\definecolor{fullyGuided}{rgb}{ 0.4275    0.6902    0.3176}
\newcommand{\fullyGuidedColor}{(dark-green)}

\begin{frontmatter}
 

%% Title, authors and addresses

%% use the tnoteref command within \title for footnotes;
%% use the tnotetext command for the associated footnote;
%% use the fnref command within \author or \address for footnotes;
%% use the fntext command for the associated footnote;
%% use the corref command within \author for corresponding author footnotes;
%% use the cortext command for the associated footnote;
%% use the ead command for the email address,
%% and the form \ead[url] for the home page:
%%
%% \title{Title\tnoteref{label1}}
%% \tnotetext[label1]{}
%% \author{Name\corref{cor1}\fnref{label2}}
%% \ead{email address}
%% \ead[url]{home page}
%% \fntext[label2]{}
%% \cortext[cor1]{}
%% \address{Address\fnref{label3}}
%% \fntext[label3]{}

\title{%\LARGE \bf
Unfinished Survey of an unbounded topic related to Breast Ultrasound images
}
\author[udg,ub]{Joan Massich}
\ead{jmassich@eia.udg.edu}
\author[udg]{Joan Mart\'{i}}
\author[ub]{Fabrice Meriaudeau}


%% use optional labels to link authors explicitly to addresses:
%% \author[label1,label2]{<author name>}
%% \address[label1]{<address>}
%% \address[label2]{<address>}

\address[udg]{Computer Vision and Robotics Group, Universitat de Girona, Campus Montilivi, Edifici PIV, s/n, 17071 Girona, Spain}
\address[ub]{Le2i-UMR CNRS 6306, Université de Bourgogne, 12 rue de la Fonderie, 712000 Le Creusot, France}

\begin{abstract}
%% Text of abstract

\end{abstract}

\begin{keyword}
%% keywords here, in the form: keyword \sep keyword

%% MSC codes here, in the form: \MSC code \sep code
%% or \MSC[2008] code \sep code (2000 is the default)

\end{keyword}

\end{frontmatter}

 \linenumbers

\graphicspath{{./figures/}}

\begin{figure}
\begin{center}
{\scalefont{0.45}

%\centering

\begin{tikzpicture}[scale=.6]

\def\labels{{\color{semiAuto}\cite{AlemanFlores:2007p14310}},
						{\color{semiAuto}\cite{Gao:2012p14336}},
						{\color{fullyAuto}\cite{Liu:2010p14328}},
{\color{semiAuto}\cite{Cui:2009p14325}},						
{\color{autoGuided}\cite{Huang:2007p6100}},
{\color{fullyAuto}\cite{Huang:2005p11636}},
{\color{semiAuto}\cite{Gomez:2010p14339}},
{\color{semiAuto}\cite{Horsch:2001p6028}},
{\color{fullyAuto}\cite{Yeh:2009p11985}},
{\color{autoGuided}\cite{Shan:2012p14347}},
{\color{autoGuided}\cite{massich2010lesion}},
{\color{semiAuto}\cite{Xiao:2002p5639,gerard2013}},
{\color{autoGuided}\cite{Zhang:2010p14317}},
{\color{fullyAuto}\cite{hao2012combining}},
{\color{autoGuided}\cite{Madabhushi:2003p6036}},
{\color{fullyAuto}\cite{Huang:2012p14313}}}

\def\reward{88.3,86.3,88.1,74.5,77.6,78.6,85.0,73.0,73.3,83.1,64.0,54.9,84.0,75.0,62.0,85.2}
\def\dbSize{32,20,76,488,118,20,50,400,6,120,25,352,347,480,42,20}
\def\dbClass{1,1,2,3,2,1,2,3,1,2,1,3,3,3,1,1}		
\def\cZoom{3} 
\def\percentageLabelAngle{90}
\def\nbeams{16}
\pgfmathsetmacro\beamAngle{(360/\nbeams)}
\pgfmathsetmacro\halfAngle{(180/\nbeams)}
%\def\globalRotation{10}
\pgfmathsetmacro\globalRotation{\halfAngle}

% draw manual AOV results
\filldraw[blue!15!white,even odd rule] (0,0) circle [radius={\cZoom*.852}] (0,0) circle [radius={\cZoom*.8}];
\draw[thin,color=blue!50!white,dashed] (0,0) circle [radius={\cZoom*.852}] (0,0) circle [radius={\cZoom*.8}];

%\foreach \x in {.125,.25, ...,1} { \draw[thin]  (0,0) circle [radius={2*\x}]; }
% draw the radiants
\foreach \n  [count=\ni] in \labels
{
\pgfmathsetmacro\cAngle{{(\ni*(360/\nbeams))+\globalRotation}}
\draw	(\cAngle:{\cZoom*1.15})  node[fill=white] {\n};
\draw [thin] (0,0) -- (\cAngle:{\cZoom*1}) ;

}

% draw the % rings 
\foreach \x in {12.5,25, ...,100} 
\draw [thin,color=gray!50] (0,0) circle [radius={\cZoom*\x/100}];

\foreach \x in {50,75,100}
{ 
     \draw [thin,color=black!50] (0,0) circle [radius={\cZoom/100*\x}];
     \foreach \a in {0, 180} \draw ({\percentageLabelAngle+\a}:{\cZoom*0.01*\x}) node  [inner sep=0pt,outer sep=0pt,fill=white,font=\fontsize{3}{3.5}\selectfont]{$\x\%$};
}





% draw the path of the percentages
\def\aux{{\reward}}
\pgfmathsetmacro\origin{\aux[\nbeams-1]} 
\draw [blue, thick] (\globalRotation:{\cZoom*\origin/100}) \foreach \n  [count=\ni] in \reward { -- ({(\ni*(360/\nbeams))+\globalRotation}:{\cZoom*\n/100}) } ;

% label all the percentags
\foreach \n [count=\ni] in \dbSize 
{
	\pgfmathsetmacro\cAngle{{(\ni*(360/\nbeams))+\globalRotation}}
	\pgfmathsetmacro\nreward{\aux[\ni-1]}
	\draw (\cAngle:{\cZoom*1.4}) node[align=center] {{\color{blue}\nreward $\%$} \\ {\color{red}\n} };
} ;

% draw the database rose
\def\dbScale{\9}
\foreach \n [count=\ni] in \dbClass
\filldraw[fill=red!20!white, draw=red!50!black]
(0,0) -- ({\ni*(360/\nbeams)-\halfAngle+\globalRotation}:{\cZoom*\n/9}) arc ({\ni*(360/\nbeams)-\halfAngle+\globalRotation}:{\ni*(360/\nbeams)+\halfAngle+\globalRotation}:{\cZoom*\n/9}) -- cycle;
\foreach \x in {1,2,3}
\draw [thin,color=red!50!black,dashed] (0,0) circle [radius={\cZoom*\x/9}];

%% draw the domain of each class 
  \def\puta{	3/0/{ACM},
  			3/3/{ACM+Other},
  			3/6/{Other}}
\def\putaa{  	2/9/{Other+ML},
  			3/11/{ML},
  			2/14/{ML+ACM}}

\foreach \numElm/\contadorQueNoSeCalcular/\name [count=\ni] in \puta
 {

 	\pgfmathsetmacro\initialAngle{(\contadorQueNoSeCalcular*\beamAngle)+\halfAngle+\globalRotation}
 	\pgfmathsetmacro\finalAngle  {((\numElm+\contadorQueNoSeCalcular)*\beamAngle)+\halfAngle+\globalRotation}
	\pgfmathsetmacro\l  {\cZoom*1.5+.3pt}
	\draw (\initialAngle:{\cZoom*1.6}) -- (\initialAngle:{\cZoom*1.1});
	\draw [ |<->|,>=latex] (\initialAngle:\l) arc (\initialAngle:\finalAngle:\l) ;    									 
	\pgfmathsetmacro\r  {\cZoom*1.5+.45pt}
    	{\draw [decoration={text along path,  text={\name},text align={center}},decorate] (\finalAngle:\r) arc (\finalAngle:\initialAngle:\r);}
  }
  
   \foreach \numElm/\contadorQueNoSeCalcular/\name [count=\ni] in \putaa
 {

 	\pgfmathsetmacro\initialAngle{(\contadorQueNoSeCalcular*\beamAngle)+\halfAngle+\globalRotation}
 	\pgfmathsetmacro\finalAngle  {((\numElm+\contadorQueNoSeCalcular)*\beamAngle)+\halfAngle+\globalRotation}
	\pgfmathsetmacro\l  {\cZoom*1.5+.3pt}
	\draw (\initialAngle:{\cZoom*1.6}) -- (\initialAngle:{\cZoom*1.1});
	\draw [ |<->|,>=latex] (\initialAngle:\l) arc (\initialAngle:\finalAngle:\l) ;    									 
	\pgfmathsetmacro\r  {\cZoom*1.5+.7pt}
    	{\draw [decoration={text along path, text={\name},text align={center}},decorate] (\initialAngle:\r) arc (\initialAngle:\finalAngle:\r);}    			 
  }
        
\end{tikzpicture}
}
\end{center}
\centering

\caption{The coloring of the reference indicates the user interacthability: semi-automatic {\color{semiAuto}\semiAutoColor}, auto-guided{\color{autoGuided}\autoGuidedColor}, and fully automatic{\color{fullyAuto}\fullyAutoColor}.} \label{fig:performanceComparison}
\end{figure}



%% The Appendices part is started with the command \appendix;
%% appendix sections are then done as normal sections
%% \appendix

%% \section{}
%% \label{}

%% References
%%
%% Following citation commands can be used in the body text:
%%
%%  \citet{key}  ==>>  Jones et al. (1990)
%%  \citep{key}  ==>>  (Jones et al., 1990)
%%
%% Multiple citations as normal:
%% \citep{key1,key2}         ==>> (Jones et al., 1990; Smith, 1989)
%%                            or  (Jones et al., 1990, 1991)
%%                            or  (Jones et al., 1990a,b)
%% \cite{key} is the equivalent of \citet{key} in author-year mode
%%
%% Full author lists may be forced with \citet* or \citep*, e.g.
%%   \citep*{key}            ==>> (Jones, Baker, and Williams, 1990)
%%
%% Optional notes as:
%%   \citep[chap. 2]{key}    ==>> (Jones et al., 1990, chap. 2)
%%   \citep[e.g.,][]{key}    ==>> (e.g., Jones et al., 1990)
%%   \citep[see][pg. 34]{key}==>> (see Jones et al., 1990, pg. 34)
%%  (Note: in standard LaTeX, only one note is allowed, after the ref.
%%   Here, one note is like the standard, two make pre- and post-notes.)
%%
%%   \citealt{key}          ==>> Jones et al. 1990
%%   \citealt*{key}         ==>> Jones, Baker, and Williams 1990
%%   \citealp{key}          ==>> Jones et al., 1990
%%   \citealp*{key}         ==>> Jones, Baker, and Williams, 1990
%%
%% Additional citation possibilities
%%   \citeauthor{key}       ==>> Jones et al.
%%   \citeauthor*{key}      ==>> Jones, Baker, and Williams
%%   \citeyear{key}         ==>> 1990
%%   \citeyearpar{key}      ==>> (1990)
%%   \citetext{priv. comm.} ==>> (priv. comm.)
%%   \citenum{key}          ==>> 11 [non-superscripted]
%% Note: full author lists depends on whether the bib style supports them;
%%       if not, the abbreviated list is printed even when full requested.
%%
%% For names like della Robbia at the start of a sentence, use
%%   \Citet{dRob98}         ==>> Della Robbia (1998)
%%   \Citep{dRob98}         ==>> (Della Robbia, 1998)
%%   \Citeauthor{dRob98}    ==>> Della Robbia


\section{references}
%% References with bibTeX database:
\bibliographystyle{model2-names}
%%\bibliographystyle{plain}
\bibliography{bibliografiaNew.bib}

%% Authors are advised to submit their bibtex database files. They are
%% requested to list a bibtex style file in the manuscript if they do
%% not want to use model2-names.bst.

%% References without bibTeX database:

% \begin{thebibliography}{00}

%% \bibitem must have one of the following forms:
%%   \bibitem[Jones et al.(1990)]{key}...
%%   \bibitem[Jones et al.(1990)Jones, Baker, and Williams]{key}...
%%   \bibitem[Jones et al., 1990]{key}...
%%   \bibitem[\protect\citeauthoryear{Jones, Baker, and Williams}{Jones
%%       et al.}{1990}]{key}...
%%   \bibitem[\protect\citeauthoryear{Jones et al.}{1990}]{key}...
%%   \bibitem[\protect\astroncite{Jones et al.}{1990}]{key}...
%%   \bibitem[\protect\citename{Jones et al., }1990]{key}...
%%   \harvarditem[Jones et al.]{Jones, Baker, and Williams}{1990}{key}...
%%

% \bibitem[ ()]{}

% \end{thebibliography}

\end{document}

%%
%% End of file `elsarticle-template-2-harv.tex'.
