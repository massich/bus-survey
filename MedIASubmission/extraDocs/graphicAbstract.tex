%\documentclass[handout]{beamer}
\documentclass[]{beamer}

%% Packages load and configuration
\usepackage 	[	style=numeric-comp,
			            	autocite=superscript, % apply options here.
			            	sortcites,
			            	maxnames=2,
				            backend=bibtex
				        ]{biblatex}
\usepackage{scalefnt,lmodern}
\usepackage{caption}
\usepackage[nolist]{acronym}
\usepackage{tikz,xifthen}
\usepackage{epsf,graphicx,subfig} %If you want to include postscript graphics 
\usepackage{multirow}
\usepackage[firstyear=2009,lastyear=2014]{moderntimeline}
%\usepackage{setspace} %to manipulate the spacing (TableOfContents were too long)  setspace redefines \@footnotetext. You could save that command in your own macro and restore it after loading setspace:
\makeatletter 
\let\std@footnotetext\@footnotetext
\usepackage{setspace}
\let\@footnotetext\std@footnotetext
\makeatother

%%clean this shit
\usepackage{amssymb, amsmath} %Some extra symbols + mathcommands
\usepackage{makeidx} %If you want to generate an index, automatically 
\usepackage{color}
\usepackage{pgfplotstable,booktabs,array,colortbl,siunitx} %thats for loading tables directly... pgfplotstable
\usepackage{rotating}% In order to have landspace tables http://tex.stackexchange.com/questions/19017/how-to-place-a-table-on-a-new-page-with-landscape-orientation-without-clearing-t
%\usepackage{mystyle} %Create your own file, mystyle.sty where you put all your own \newcommand statements, for example. 
\usepackage{pstricks}
\usepackage{scalefnt,lmodern}
\usepackage{nicefrac}
%\usepackage{tabularx}
\usepackage{bitset}
\usepackage{epigraph}
\usepackage{pgfplots}
\usepackage{textcomp} 
\usepackage{layout}
%%end clean

%% Extra configuration to use the timeline in beamer
\makeatletter
% change these colors according to your needs
\colorlet{color0}{blue}
\colorlet{color1}{olive}

\newcommand*{\hintfont}{}
\newcommand*{\hintstyle}[1]{{\hintfont\textcolor{color0}{#1}}}
\newcommand*{\listitemsymbol}{a~}
\newcommand*{\cventry}[7][.25em]{%
  \cvitem[#1]{#2}{%
    {\bfseries\raggedright #3}%
    \ifthenelse{\equal{#4}{}}{}{, \raggedright{\slshape#4}}%
    \ifthenelse{\equal{#5}{}}{}{,  \raggedright#5}%
    \ifthenelse{\equal{#6}{}}{}{, \raggedright#6}%
    .\strut%
    \ifx&#7&%
      \else{\newline{}\begin{minipage}[t]{\linewidth}\small\raggedright#7\end{minipage}}\fi}}
\newcommand*{\cvitem}[3][.25em]{%
  \begin{tabular}{@{}p{\hintscolumnwidth}@{\hspace{\separatorcolumnwidth}}p{\maincolumnwidth}@{}}%
      \raggedleft\hintstyle{#2} & {#3}%
  \end{tabular}%
  \par\addvspace{#1}}
\tlmaxdates{2009}{2014}
\newlength{\quotewidth}
\newlength{\hintscolumnwidth}
\setlength{\hintscolumnwidth}{0.175\textwidth}
\newlength{\separatorcolumnwidth}
\setlength{\separatorcolumnwidth}{0.025\textwidth}
\newlength{\maincolumnwidth}
\newlength{\doubleitemmaincolumnwidth}
\newlength{\listitemsymbolwidth}
\settowidth{\listitemsymbolwidth}{\listitemsymbol}
\newlength{\listitemmaincolumnwidth}
\newlength{\listdoubleitemmaincolumnwidth}
\setlength{\maincolumnwidth}{\dimexpr0.9\linewidth-\separatorcolumnwidth-\hintscolumnwidth\relax}
\tikzset{
    tl@startyear/.append style={
        xshift=(0.5-\tl@startfraction)*\hintscolumnwidth,
        anchor=base
    }
}
\makeatother


%% Extra packages configuration
\addbibresource{bibliografiaNew.bib}
\captionsetup{justification={justified}}
\DeclareGraphicsExtensions{.pdf,.jpeg,.png}
\usetikzlibrary{arrows,calc,positioning,fit,backgrounds,snakes,decorations,decorations.text,decorations.markings,shapes,patterns}

% packages for the iconography of the table
\usepackage{stackengine}
%\usepackage{xcolor}
\usepackage{bbding}
\usepackage{scalerel}
\usepackage{ifthen}

%%----- To generate stand onle tikz legends
% argument #1: any options
\newenvironment{customlegend}[1][]{%
    \begingroup
    % inits/clears the lists (which might be populated from previous
    % axes):
    \csname pgfplots@init@cleared@structures\endcsname
    \pgfplotsset{#1}%
}{%
    % draws the legend:
    \csname pgfplots@createlegend\endcsname
    \endgroup
}%
% makes \addlegendimage available (typically only available within an
% axis environment):
\def\addlegendimage{\csname pgfplots@addlegendimage\endcsname}
\pgfkeys{/pgfplots/number in legend/.style={%
        /pgfplots/legend image code/.code={%
            \node at (0.295,-0.0225){#1};
        },%
    },
}
%%---- end tikz legends

% guilloume
\newenvironment<>{varblock}[2][.9\textwidth]{%
\setlength{\textwidth}{#1}
\begin{actionenv}#3%
\def\insertblocktitle{#2}%
\par%
\usebeamertemplate{block begin}}
{\par%
\usebeamertemplate{block end}%
\end{actionenv}}


%% Overwrite beamer options
\setbeamertemplate{navigation symbols}{}
\usetheme{Warsaw}
\beamersetuncovermixins{\opaqueness<1>{25}}{\opaqueness<2->{15}}
\setbeamerfont{footnote}{size=\tiny}
\setbeamerfont{caption}{size=\scriptsize}

\newcounter{ii}
%\includeonly{chapters/introduction/intro}
%\includeonly{chapters/survey/survey}
%\includeonly{chapters/method/method}
%\includeonly{dummy}
%\includeonly{lexicon}

\newcommand*{\ch}{\checkmark}
\begin{document}

\definecolor{autoGuided}{rgb}{ 0.3765    0.7294    0.9412}
\newcommand{\autoGuidedColor}{(light-Blue)}
\definecolor{fullyAuto}{rgb}{ 0.0941    0.3843    0.6627}
\newcommand{\fullyAutoColor}{(dark-blue)}
\definecolor{semiAuto}{rgb}{ 0.0784    0.5059    0.1686}
\newcommand{\semiAutoColor}{(light-green)}
\definecolor{fullyGuided}{rgb}{ 0.4275    0.6902    0.3176}
\newcommand{\fullyGuidedColor}{(dark-green)}
\definecolor{colorTheme}{RGB}{51,41,178}



\title{Segmentation of breast lesions in Ultrasound images: A survey}  
%\author[Joan Massich]{Joan Massich Vall \\ \vspace{5pt} {\footnotesize \begin{tabular}{rl}
%Supervised by: & Joan Mart\'{i} Bonmat\'{i} \\
%& Fabrice Meriaudeau
%\end{tabular}}}



\titlegraphic{
 \centering
 \vspace{-2cm}
\includegraphics[height = 1 cm]{Logo_UdG.eps}\hspace*{6cm}~%
   \noindent \includegraphics[height = 1 cm]{uB.eps}
}
\date{December 13, 2013} 
\institute[UdG, uB]{
%Universitat de Girona, Universit\'{e} de Bourgogne\\
\medskip {\emph{fabrice.meriaudeau@u-bourgogne.fr}}
}
%\date{\today}

\graphicspath{{generalFigures/}}

\begin{frame} \frametitle{Reported results in terms of area overlap (AOV) coefficient}
\vspace{-5pt}
\begin{figure}
\tiny
\setlength{\abovecaptionskip}{2pt}
\centering
\begin{tikzpicture}[scale=.58]
\def\labels{	{\color{semiAuto}[68]},%\cite{AlemanFlores:2007p14310}},
						{\color{semiAuto}[70]},%\cite{Gao:2012p14336}},
						{\color{fullyAuto}[85]},%\cite{Liu:2010p14328}},
						{\color{semiAuto}[69]},%\cite{Cui:2009p14325}},					
						{\color{autoGuided}[75]},%\cite{Huang:2007p6100}},
						{\color{fullyAuto}[80]},%\cite{Huang:2005p11636}},
						{\color{semiAuto}[64]},%\cite{Gomez:2010p14339}},
						{\color{semiAuto}[58]},%\cite{Horsch:2001p6028}},
						{\color{fullyAuto}[86]},%\cite{Yeh:2009p11985}},
						{\color{autoGuided}[79]},%\cite{Shan:2012p14347}},
						{\color{autoGuided}[61]},%\cite{massich2010lesion}},
						{\color{semiAuto}[66]},%\cite{Xiao:2002p5639,gerard2013}},
						{\color{autoGuided}[76]},%\cite{Zhang:2010p14317}},
						{\color{fullyAuto}[84]},%\cite{hao2012combining}},
						{\color{autoGuided}[60]},%\cite{Madabhushi:2003p6036}},
						{\color{fullyAuto}[81]}}%\cite{Huang:2012p14313}}}

\def\reward{88.3,86.3,88.1,74.5,77.6,78.6,85.0,73.0,73.3,83.1,64.0,54.9,84.0,75.0,62.0,85.2}
\def\dbSize{32,20,76,488,118,20,50,400,6,120,25,352,347,480,42,20}
\def\dbClass{1,1,2,3,2,1,2,3,1,2,1,3,3,3,1,1}		
\def\cZoom{3} 
\def\percentageLabelAngle{90}
\def\nbeams{16}
\pgfmathsetmacro\beamAngle{(360/\nbeams)}
\pgfmathsetmacro\halfAngle{(180/\nbeams)}
%\def\globalRotation{10}
\pgfmathsetmacro\globalRotation{\halfAngle}

% draw manual AOV results
\filldraw[blue!15!white,even odd rule] (0,0) circle [radius={\cZoom*.852}] (0,0) circle [radius={\cZoom*.8}];
\draw[thin,color=blue!50!white,dashed] (0,0) circle [radius={\cZoom*.852}] (0,0) circle [radius={\cZoom*.8}];

%\foreach \x in {.125,.25, ...,1} { \draw[thin]  (0,0) circle [radius={2*\x}]; }
% draw the radiants with the reference label
\foreach \n  [count=\ni] in \labels
{
\pgfmathsetmacro\cAngle{{(\ni*(360/\nbeams))+\globalRotation}}
\draw [thin] (0,0) -- (\cAngle:{\cZoom*1}) ;
\draw	(\cAngle:{\cZoom*1.1})  node[fill=white, inner sep=0pt] {{\tiny \textbf  \n}}; %referencies
}


% draw the % rings 
\foreach \x in {12.5,25, ...,100} 
\draw [thin,color=gray!50] (0,0) circle [radius={\cZoom*\x/100}];

\foreach \x in {50,75,100}
{ 
     \draw [thin,color=black!50] (0,0) circle [radius={\cZoom/100*\x}];
     \foreach \a in {0, 180} \draw ({\percentageLabelAngle+\a}:{\cZoom*0.01*\x}) node  [inner sep=0pt,outer sep=0pt,fill=white,font=\fontsize{5}{5}\selectfont]{$\x$};
}



% draw the path of the percentages
\def\aux{{\reward}}
\pgfmathsetmacro\origin{\aux[\nbeams-1]} 
\draw [blue, thick] (\globalRotation:{\cZoom*\origin/100}) \foreach \n  [count=\ni] in \reward { -- ({(\ni*(360/\nbeams))+\globalRotation}:{\cZoom*\n/100}) } ;

% label all the percentags
\foreach \n [count=\ni] in \dbSize 
{
	\pgfmathsetmacro\cAngle{{(\ni*(360/\nbeams))+\globalRotation}}
	\pgfmathsetmacro\nreward{\aux[\ni-1]}
	\draw (\cAngle:{\cZoom*1.4}) node[align=center] {{\color{blue}\nreward $\%$} \\ {\color{red}\n} };
} ;

% draw the database rose
\def\dbScale{\9}
\foreach \n [count=\ni] in \dbClass
\filldraw[fill=red!20!white, draw=red!50!black]
(0,0) -- ({\ni*(360/\nbeams)-\halfAngle+\globalRotation}:{\cZoom*\n/9}) arc ({\ni*(360/\nbeams)-\halfAngle+\globalRotation}:{\ni*(360/\nbeams)+\halfAngle+\globalRotation}:{\cZoom*\n/9}) -- cycle;
\foreach \x in {1,2,3}
\draw [thin,color=red!50!black,dashed] (0,0) circle [radius={\cZoom*\x/9}];

%% draw the domain of each class 
  \def\puta{	3/0/{ACM},
  			3/3/{ACM+Other},
  			3/6/{Other}}
\def\putaa{  	2/9/{Other+ML},
  			3/11/{ML},
  			2/14/{ML+ACM}}

\foreach \numElm/\contadorQueNoSeCalcular/\name [count=\ni] in \puta
 {

 	\pgfmathsetmacro\initialAngle{(\contadorQueNoSeCalcular*\beamAngle)+\halfAngle+\globalRotation}
 	\pgfmathsetmacro\finalAngle  {((\numElm+\contadorQueNoSeCalcular)*\beamAngle)+\halfAngle+\globalRotation}
	\pgfmathsetmacro\l  {\cZoom*1.5+.3pt}
	\draw (\initialAngle:{\cZoom*1.6}) -- (\initialAngle:{\cZoom*1.1});
	\draw [ |<->|,>=latex] (\initialAngle:\l) arc (\initialAngle:\finalAngle:\l) ;    									 
	\pgfmathsetmacro\r  {\cZoom*1.5+.45pt}
    	{\draw [decoration={text along path,  text={\name},text align={center}},decorate] (\finalAngle:\r) arc (\finalAngle:\initialAngle:\r);}
  }
  
   \foreach \numElm/\contadorQueNoSeCalcular/\name [count=\ni] in \putaa
 {

 	\pgfmathsetmacro\initialAngle{(\contadorQueNoSeCalcular*\beamAngle)+\halfAngle+\globalRotation}
 	\pgfmathsetmacro\finalAngle  {((\numElm+\contadorQueNoSeCalcular)*\beamAngle)+\halfAngle+\globalRotation}
	\pgfmathsetmacro\l  {\cZoom*1.5+.3pt}
	\draw (\initialAngle:{\cZoom*1.6}) -- (\initialAngle:{\cZoom*1.1});
	\draw [ |<->|,>=latex] (\initialAngle:\l) arc (\initialAngle:\finalAngle:\l) ;    									 
	\pgfmathsetmacro\r  {\cZoom*1.5+.7pt}
    	{\draw [decoration={text along path, text={\name},text align={center}},decorate] (\initialAngle:\r) arc (\initialAngle:\finalAngle:\r);}    			 
  }
  
    \node [anchor=north west] at (\cZoom*1.8,\cZoom*1.5){
	\begin{tikzpicture}
  \begin{customlegend}[legend entries={Dataset size categorization,Dataset size,AOV results, Manual delineation AOV, AOV percentage}]
    \addlegendimage{red,fill=red!20!white, draw=red!60!gray, ybar, ybar legend}
    \addlegendimage{number in legend=XX,red}
    \addlegendimage{red,draw=blue,thick,sharp plot}
    \addlegendimage{red,fill=blue!15!white,draw=blue,dashed,area legend}
    \addlegendimage{number in legend=XX.X\%,blue}

    \end{customlegend}
\end{tikzpicture}}; 

\end{tikzpicture}
\caption{{\tiny Circular representation of the reported results grouped by methodology class. From inside to outside: used dataset categorization, AOV, work ID, numeric data and methodology category label. The work ID is colored as follows: semi-automatic {\color{semiAuto}\semiAutoColor}, auto-guided{\color{autoGuided}\autoGuidedColor}, and fully automatic{\color{fullyAuto}\fullyAutoColor}.}} 
\end{figure}
\end{frame}

\end{document}
