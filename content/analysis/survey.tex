% include the figures path relative to the master file
\graphicspath{ {./content/analysis/figures/} }

% \section{Segmentation method analysis}
\section{Segmentation}
\label{sec:seg-analysis}
\todo[inline, caption={segmentation story}]{
  \begin{itemize}
    \item Describe segmentation
    \item medical imaging particularity: interaction
    \item Segmentation 'technology'
      \begin{itemize}
        \item analogy to regular segmentation
          \begin{itemize}
            \item chronological: region, boundary, mix
            \item descriptions: from others to my description
          \end{itemize}
        \item my description
      \end{itemize}
    \item Segmentation 'interaction'
      \begin{itemize}
        \item summary
        \item foundings
      \end{itemize}
  \end{itemize}
}

Taking xxxxxxx as a reference, the process of segmentation consists of xxxx.

With all these assumptions the problem can be approached by finding the classes or the interfaces between these classes, leading to region- or boundary-based approaches respectively.

, and Chronologically, the most straight forward solution 

\subsection{Segmentation technology}
\subsection{Segmentation interaction}
