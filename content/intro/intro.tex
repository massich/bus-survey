% include the figures path relative to the master file
\graphicspath{ {./content/intro/figures/} }

\section*{Introduction}
\label{sec:intro}  % \label{} allows reference to this section
\todo[inline, caption={Stuff to cover in the intro}]{
  \begin{itemize}
    \item Narrow to the need of accurate delineations
    \item What is reviewed and what is not
    \item Article Objectives
    \item similar works
    \item paper structure
  \end{itemize}
}

\paragraph{the need of accurate delineations}
\label{sec:intro:to_delinations}  % \label{} allows reference to this section
\todo[inline, caption={Narrow to need of accurate delineations}]{
  \begin{enumerate}
      \item breast cancer kills
      \item screening is needed for early detection
      \item Health from images is like any other visual inspection
        \begin{itemize}
            \item Radiologic diagnisis error rates are similar to any other human visual inspection\cite{manning2005perception}
            \item Utilization of computers to aid the Radiologists during the diagnosis process\cite{giger2008anniversary}
        \end{itemize}
      \item Advantage of \us
      \item \birads
      \item need of accurate delineations
  %     % \item
  \end{enumerate}
}
Breast cancer is the leading cause of cancer deaths among females worldwide~\cite{cancerStatistics2011}.
Nevertheless, death by breast cancer are highly reduced when early treated.
Thus, to run a chance of surviving breast cancer, it is uttermost important the early detection of malignant tumors.
This has motivated the establishment of \acp{bsp} to facilitate this breast cancer detection at an early stage.
Despite X-ray \dm is considered the gold standard technique for \bsp, other screening techniques like \us and \mri are being investigated
to overcome \dm limitations due to tissue superposition which can either mimic or obscure malignant pathology,
and avoid X-ray radiation all together.


Medical imaging contributes to its early detection through screening programs, non-invasive diagnosis, follow-up, and similar procedures.
Despite \ac{bus} imaging not being the imaging modality of reference for breast cancer screening~\cite{smith2003american}, \ac{us} imaging has more discriminative power when compared with other image modalities to visually differentiate benign from malignant solid lesions~\cite{Stavros:1995p12392}.
In this manner, \ac{us} screening is estimated to be able to reduce
between $65\sim85\%$ of unnecessary biopsies%~\cite{yuan2010multimodality}
, in favour of a less traumatic short-term screening follow-up using \ac{bus} images.
As the standard for assessing these \ac{bus} images, the \ac{acr} proposes the \ac{birads} lexicon for \ac{bus} images~\cite{biradsus}.
This \ac{us} \ac{birads} lexicon is a set of standard markers that characterizes the lesions encoding the visual cues found in \ac{bus} images and facilitates their analysis.
Further details regarding the \ac{us} \ac{birads} lexicon descriptors proposed by the \ac{acr}, can be found in Sect.\,\ref{sec:methodApp}, where visual cues of \ac{bus} images and breast structures are discussed to define feature descriptors.

The incorporation of \ac{us} in screening policies and the emergence of clinical standards to assess image like the \ac{us} \ac{birads} lexicon, encourage the development of \ac{cad} systems using \ac{us} to be applied to breast cancer diagnosis.
However, this clinical assessment using lexicon is not directly applicable to \ac{cad} systems.
Shortcomings like the location and explicit delineation of the lesions need to be addressed, since those tasks are intrinsically carried out by the radiologists during their visual assessment of the images to infer the lexicon representation of the lesions.

\paragraph{What is reviewed}
\label{sec:intro:what_is_reviewed}

\paragraph{Article objective}
\label{sec:intro:article_objective}

The objective of this article is
\begin{itemize}
  \item \added[id=sik]{to provide an exhaustive list of segmentation methodologies that have been developed for delineating breast lesions in \us images.}
  \item \added[id=sik]{to collect a set of terms that facilitate placing each work with respect of the rest of the \stateArt.}
  \item \added[id=sik]{to provide an overview of how each methodology has been assessed.}
  \item \added[id=sik]{to clarify assessment assumptions that influence a fair results comparison.}
\end{itemize}

A secondary objective of the authors, but equally essential to us, is
\begin{itemize}
  \item \added[id=sik]{to ensure comprehensive coverage of the available segmentation methodologies.}
  \item \added[id=sik]{fair treatment.}
  \item \added[id=sik]{reusability of the efforts put in this article.}
\end{itemize}


\paragraph{similar works}
\label{sec:intro:similar_works}

\paragraph{Paper structure}
\label{sec:intro:paper_structure}


% Some stuff that emac's colegues use
%%% Local Variables:
%%% mode: late
%%% TeX-master: "../../main.tex"
%%% End: \section{introduction}

